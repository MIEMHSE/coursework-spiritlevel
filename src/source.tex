\documentclass[document.tex]{subfiles}
\begin{document}
\section{Исходные коды}
\subsection{Общая концепция программной части}
Общая концепция программной части подразумевает подключение модулей классов и дальнейшее использование их в виде объектов. Для каждого из объектов в бесконечном цикле вызывается
метод loop, который определяется в каждом из классов и предназначен для выполнения действий, проходящих за одну итерацию глобального цикла. В этот метод в виде аргументов
передается глобальное состояние~--- ассоциативный массив пар ключ (наименование класса) -- значение (ассоциативный массив, возвращаемый переменной state после завершения процедуры
loop). Код основного модуля с бесконечным циклом, который предварительно импортирует классы получения значений с акселерометра (AccelInfo), вывода значений на дисплей (AccelPlot) и
изменения состояния светодиодов (LEDAccel), пред\-ставлен в листинге~\ref{lst:spiritlevel}.

\begin{listing}[ht]
\begin{minted}[linenos=true]{python}
__author__ = 'Sergey Sobko'

import pyb

# Подключаем необходимые модули
from plot import AccelPlot
from led import LEDAccel
from accel import AccelInfo

# Для каждого импортированного класса создаем объекты
board_classes = [AccelInfo, AccelPlot, LEDAccel]
board_objects = [board_class() for board_class in board_classes]

# Создаем пустой ассоциативный массив глобального состояния
global_state = dict()

# Бесконечный цикл
while True:
    
    # Для каждого из объектов
    for board_object in board_objects:

        # Вызываем метод loop с передачей ассоциативного массива 
        # глобального состояния
        board_object.loop(**global_state)
        
        # Обновляем ассоциативный массив глобального состояния
        # значениями из свойства state объекта
        global_state.update({
            board_object.__class__.__name__: board_object.state
        })
\end{minted}
\caption{spiritlevel.py -- основной модуль с бесконечным циклом}
\label{lst:spiritlevel}
\end{listing}

\clearpage
\subsection{Получение значений с акселерометра}

Код класса получения значений с акселерометра представлен в листинге~\ref{lst:accel}.

\begin{listing}[ht]
\begin{minted}[linenos=true]{python}
__author__ = 'Sergey Sobko'

import pyb

INIT_DELAY = 20  # Задержка для калибровки
LOOP_DELAY = 0  # Задержка между итерациями


# Класс для получения значений акселерометра
class AccelInfo(object):
    x, y, z = None, None, None
    tilt = None

    # Инициализация класса акселерометра
    def __init__(self):
        self.accel = pyb.Accel()  # Инициализация аппаратного модуля
        pyb.delay(INIT_DELAY)  # Задержка для калибровки

    # Итерация цикла
    def loop(self, **kwargs):
        # Получаем фильтрованные значения (от -90 до 90) по трем осям
        self.x, self.y, self.z = self.accel.filtered_xyz()
        
        # Получаем состояния регистра встряхивания
        self.tilt = self.accel.tilt()
        
        # Задерживаем исполнение, если предусмотрена задержка цикла
        if LOOP_DELAY:
            pyb.delay(LOOP_DELAY)

    # Получение значений отклонения по трем осям и регистра встряхивания
    @property
    def state(self):
        return {
            'x': self.x,
            'y': self.y,
            'z': self.z,
            'tilt': self.tilt
        }
\end{minted}
\caption{accel/\_\_init\_\_.py -- модуль получения значений с акселерометра}
\label{lst:accel}
\end{listing}

\clearpage
\subsection{Управление светодиодами}

Код класса управления светодиодами представлен в листинге~\ref{lst:led}.

\begin{listing}[ht]
\begin{minted}[linenos=true]{python}
__author__ = 'Sergey Sobko'

import pyb

ACCEL_EPS_X = 30
ACCEL_EPS_Y = 30
ACCEL_EPS_TILT = 50


class LEDAccel(object):
    def __init__(self):
        self.led = pyb.LED(3)
        self.led_tilt = pyb.LED(2)
        self.on = False
        self.on_tilt_counter = 0

    def loop(self, **kwargs):
        accel_info = kwargs.get('AccelInfo')
        self.on = False
        if accel_info:  # Есть данные с акселерометра
            x, y, z, tilt = map(accel_info.get, ['x', 'y', 'z', 'tilt'])
            if abs(x) > ACCEL_EPS_X or abs(y) > ACCEL_EPS_Y:
                self.on = True
            if tilt > ACCEL_EPS_TILT:
                self.on_tilt_counter = 10
                self.led_tilt.on()

        # Проверка кол-ва прошедших тактов светодиода регистра встряхивания
        if self.on_tilt_counter > 0:
            self.on_tilt_counter -= 1
        else:
            self.led_tilt.off()
            self.on_tilt_counter = 0

        # Если превышение максимально допустимого отклонения по X, Y, Z
        if self.on:
            self.led.on()
        else:
            self.led.off()

    @property
    def state(self):
        return dict(on=self.on, led=self.led)
\end{minted}
\caption{led/\_\_init\_\_.py -- управления светодиодами}
\label{lst:led}
\end{listing}


\clearpage
\subsection{Вывод на дисплей}
Код класса вывода информации на дисплей представлен в листинге~\ref{lst:lcd}.

\begin{listing}[ht]
\begin{minted}[linenos=true]{python}
__author__ = 'Sergey Sobko'

import pyb

LOOP_DELAY_GRAPHICS = 10
MAXX = 127.0
MAXY = 31.0
ACCEL_COEFF_X = MAXX / 90 / 2
ACCEL_COEFF_Y = MAXY / 90 / 2


class AccelPlot(Plot):
    # Инициализация и подсвечивание LCD
    def __init__(self):
        self.lcd = pyb.LCD('X')
        self.lcd.light(True) 
    
    # Вывод по точкам прямоугольника из центра LCD до (x, y)
    def draw_square_for_accel(self, x, y):
        min_x = int(min(MAXX / 2 + x * ACCEL_COEFF_X, MAXX / 2))
        max_x = int(max(MAXX / 2 + x * ACCEL_COEFF_X, MAXX / 2))
        min_y = int(min(MAXY / 2 - y * ACCEL_COEFF_Y, MAXY / 2))
        max_y = int(max(MAXY / 2 - y * ACCEL_COEFF_Y, MAXY / 2))
        for x in range(min_x, max_x):
            self.lcd.pixel(x, min_y, 1)
            self.lcd.pixel(x, max_y, 1)
        for y in range(min_y, max_y):
            self.lcd.pixel(min_x, y, 1)
            self.lcd.pixel(max_x, y, 1)

    def loop(self, **kwargs):
        accel_info = kwargs.get('AccelInfo')
        if accel_info:  # Есть данные с акселерометра
            x, y, z = map(accel_info.get, ['x', 'y', 'z'])
            self.lcd.fill(0)
            self.draw_square_for_accel(x, y)
            if LOOP_DELAY_GRAPHICS:
                pyb.delay(LOOP_DELAY_GRAPHICS)
            self.lcd.show()
            
    @property
    def state(self):
        return dict(lcd=self.lcd)
\end{minted}
\caption{plot/\_\_init\_\_.py -- вывод информации на дисплей}
\label{lst:lcd}
\end{listing}

\end{document}
