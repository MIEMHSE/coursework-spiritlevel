\documentclass[document.tex]{subfiles} 
\begin{document}
\section{Вывод по работе и рекомендации к улучшению}
В ходе выполнения курсовой работы были рассмотрены аппаратные продукты различных фирм, изучены способы их соединения между собой с использованием стандартизованных интерфейсов
I\textsuperscript{2}C и SPI.

Реализован программно-аппаратный комплекс, позволяющий отслеживать отклонение от горизонтальной поверхности в виде прямоугольника, проводимого из центра до точки,
противонаправленной отклонению на пропорциональный количеству пикселов дисплея градус и сигнализирующий светодиодом в случае превышения заданного значения отклонения.
 
Дальнейшие работы по проекту предполагают расширение способов сигнали\-зации пользо\-вателя о превышении допустимого отклонения. Основной идеей улучшения может служить возможность
передачи данных замеров инклинометра на компьютер для централизованной обработки принимаемых данных. Такая модель поведения может быть использована, напри\-мер, для контроля
ровности заливки бетонных полов при строительстве и других применени\-ях такого инстру\-мента как уровень.

Общий прогресс разработки и само программное обеспечение электронного уровня можно получить из
системы контроля версий GitHub по следующемум адресу: \\ \indent {\url{https://github.com/MIEMHSE/spiritlevel}}.
\end{document}
