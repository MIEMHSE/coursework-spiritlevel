\documentclass[document.tex]{subfiles} 
\begin{document}
\section{Аппаратное обеспечение}
\subsection{Акселерометр}
Акселерометр может применяться как для измерения проекций абсолютного линейного
ускорения, так и для косвенных~(через силу реакции опоры) измерений проекции
гравитаци\-онного ускорения.

Первое свойство используется для создания инерциальных навигационных систем, где
полученные с помощью акселерометров измерения интегрируют, получая
инерци\-альную ско\-рость и координаты носителя. Таким образом, акселерометры, наравне с
гироско\-пами, являются неотъемлемыми компонентами систем навигации и управления
самолётов, ракет и других летательных аппаратов, кораблей и подводных лодок.
Второе свойство позволяет использовать акселерометры для измерения уклонов, то
есть в качестве инклино\-метров.\cite{accelerometer_info}

Используя акселерометр в качестве инклинометра, то есть прибора,
предназначенного для измерения величины и азимута угла наклона различных
объектов относительно гравита\-ционного поля Земли, можно реализовать функционал,
характерный для уровней.

Между цифровыми и аналоговыми акселерометрами, следудет выбрать цифровые, так
как они не требуют внешних компонентов, и не требуют никаких расчётов: все их
метро\-логические характеристики указаны. Стоить они будут дороже аналоговых, но
время, затра\-чиваемое на разработку системы
снижается.\cite{accelerometer_compare}

Общие сравнительные характеристики цифровых акселерометров указаны в таблице~\ref{tabular:accelerometer_compare_common}.
Неплохим выбором, исходя из этой таблицы, получается MMA7660 компании Freescale Semiconductor, имеющий следующие ТТХ:
\begin{itemize}
	\item цифровой вывод (I2C);
	\item 3mm x 3mm x 0.9mm DFN-чип; 
	\item низкое энергопотребление;
	\item настраиваемую частоту снятия показаний от 1 до 120 в секунду;
	\item низкий вольтаж (2.4 В -- 3.6 В~--- аналоговый; 1.71 В -- 3.6 В~--- цифровой);
	\item автоматический режим сна для уменьшения энергопотребления;
	\item определение ориентации в пространстве;
	\item совместимость с RoHS;
	\item отсутствие галогенов;
	\item низкую цену.
\end{itemize}

\noindent
Примеры типового применения акселерометра MMA7660, согласно документации, следующие:
\begin{itemize} 
	\item мобильные телефоны/ PMP/PDA: оперделение ориентации (портретная/пейзажная), стабилизация изображения, прокрутка текста, звонок по встряхиванию;
	\item настольные компьютеры: защита от кражи;
	\item игры: определение движения, автоматический режим сна для уменьшения энерго\-потребления;
	\item цифровые камеры: стабилизация изображения. \cite{accelerometer_mma7660}
\end{itemize}

\noindent
\begin{sidewaystable}
\caption{Общие характеристики цифровых акселерометров}
\label{tabular:accelerometer_compare_common}
\medskip
\resizebox{\linewidth}{!}{
\tabcolsep=2pt
\begin{tabular}{|c|c|c|c|c|c|c|c|c|}
\toprule
Модель&Количество осей&Напряжение питания&Интерфейс&
Пределы измерений&Частота выборки, Гц&Погрешность
нуля, mg&Разрешение, mg\\
\hline
MMA7450		&3		&2.4 -- 3.6~В	&I2C, SPI	&\pm2g, \pm4g, \pm8g			&125, 250	&250	&15.6 \\
MMA7660		&3		&2.4 -- 3.6~В	&I2C		&\pm1.5g						&1 -- 120	&64		&21.33 \\
MMA7455		&3		&2.4 -- 3.6~В	&I2C, SPI	&\pm2g, \pm4g, \pm8g			&125, 250	&330	&15.6 \\
ADXL345		&3		&2.0 -- 3.6~В	&I2C, SPI	&\pm2g, \pm4g, \pm8g, \pm16g	&0.1 -- 3200&150	&3.9 \\
SMB380		&3		&2.4 -- 3.6~В	&I2C, SPI	&\pm2g, \pm4g, \pm8g			&25 -- 1500	&60		&4 \\
LIS202DL	&2		&2.2 -- 3.6~В	&I2C, SPI	&\pm2g, \pm8g					&100, 400	&40		&18 \\
LSM303DLM	&$3 + 3$&2.16 -- 3.6~В	&I2C		&\pm2g, \pm4g, \pm8g			&			&60		&1 \\
\bottomrule
\end{tabular}}
\end{sidewaystable}

\end{document}
