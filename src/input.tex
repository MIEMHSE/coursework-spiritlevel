\documentclass[document.tex]{subfiles} 
\begin{document}
\section{Постановка задачи}
Уровень представляет собой измерительный инструмент прямоугольный фор\-мы из
пла\-стика, дерева или металла с установленными в нем прозрачными колбами
(глазками), запол\-ненными жидкостью. Уровень был разработан для оценки
соответствия поверхностей верти\-кальной или горизонтальной плоскости, а также
для измерения градуса отклонения поверх\-ности от горизонтальной плоскости.

Для проверки уровня (то есть оценки точности проводимых им измерений) необходимо
установить его на ровную горизонтальную поверхность и замерить положение
пузырька воздуха относительно двух рисок в центре колбы. Затем уровень нужно
перевернуть в горизонтальной плоскости на 180 градусов и произвести повторный
замер положения пузырь\-ка. Если уровень исправен, то пузырек воздуха будет точно
в том же положении, что и при первом измерении. Для регулировки
инструмента~(если это предусмотрено, колба будет закреплена регулировочными
винтами на теле уровня) необходимо попеременно вращать уровень на 180 градусов в
горизонтальной плоскости и регулировать положение глазка до тех пор, пока его
показания не будут идентичными при вращении инструмента. Для подобной
регулировки не требуется идеально горизонтальная или вертикальная
поверхность.\cite{spiritlevel_info}

Необходимо разработать программно-аппаратный комплекс бытового элек\-ронного
уров\-ня с возможностью калибровки, отображения информации откло\-нения относительно
поверх\-ностей и передачи сигнальной информации на ЭВМ.

Основная идея заключается в использовании инклинометра (акселерометра) для опреде\-ления положения комплекса относительно горизонтального, микро\-контроллера для обработки данных
и устройства отображения (дисплея) для вывода результатов, понятных для воспри\-ятия.
\end{document}
